% Options for packages loaded elsewhere
\PassOptionsToPackage{unicode}{hyperref}
\PassOptionsToPackage{hyphens}{url}
%
\documentclass[
]{article}
\usepackage{lmodern}
\usepackage{amssymb,amsmath}
\usepackage{ifxetex,ifluatex}
\ifnum 0\ifxetex 1\fi\ifluatex 1\fi=0 % if pdftex
  \usepackage[T1]{fontenc}
  \usepackage[utf8]{inputenc}
  \usepackage{textcomp} % provide euro and other symbols
\else % if luatex or xetex
  \usepackage{unicode-math}
  \defaultfontfeatures{Scale=MatchLowercase}
  \defaultfontfeatures[\rmfamily]{Ligatures=TeX,Scale=1}
\fi
% Use upquote if available, for straight quotes in verbatim environments
\IfFileExists{upquote.sty}{\usepackage{upquote}}{}
\IfFileExists{microtype.sty}{% use microtype if available
  \usepackage[]{microtype}
  \UseMicrotypeSet[protrusion]{basicmath} % disable protrusion for tt fonts
}{}
\makeatletter
\@ifundefined{KOMAClassName}{% if non-KOMA class
  \IfFileExists{parskip.sty}{%
    \usepackage{parskip}
  }{% else
    \setlength{\parindent}{0pt}
    \setlength{\parskip}{6pt plus 2pt minus 1pt}}
}{% if KOMA class
  \KOMAoptions{parskip=half}}
\makeatother
\usepackage{xcolor}
\IfFileExists{xurl.sty}{\usepackage{xurl}}{} % add URL line breaks if available
\IfFileExists{bookmark.sty}{\usepackage{bookmark}}{\usepackage{hyperref}}
\hypersetup{
  pdftitle={Tarea 3: LaTeX y RMarkdown},
  pdfauthor={\textbackslash mathcal Imanol\textbackslash{} Miguez},
  hidelinks,
  pdfcreator={LaTeX via pandoc}}
\urlstyle{same} % disable monospaced font for URLs
\usepackage[margin=1in]{geometry}
\usepackage{color}
\usepackage{fancyvrb}
\newcommand{\VerbBar}{|}
\newcommand{\VERB}{\Verb[commandchars=\\\{\}]}
\DefineVerbatimEnvironment{Highlighting}{Verbatim}{commandchars=\\\{\}}
% Add ',fontsize=\small' for more characters per line
\usepackage{framed}
\definecolor{shadecolor}{RGB}{248,248,248}
\newenvironment{Shaded}{\begin{snugshade}}{\end{snugshade}}
\newcommand{\AlertTok}[1]{\textcolor[rgb]{0.94,0.16,0.16}{#1}}
\newcommand{\AnnotationTok}[1]{\textcolor[rgb]{0.56,0.35,0.01}{\textbf{\textit{#1}}}}
\newcommand{\AttributeTok}[1]{\textcolor[rgb]{0.77,0.63,0.00}{#1}}
\newcommand{\BaseNTok}[1]{\textcolor[rgb]{0.00,0.00,0.81}{#1}}
\newcommand{\BuiltInTok}[1]{#1}
\newcommand{\CharTok}[1]{\textcolor[rgb]{0.31,0.60,0.02}{#1}}
\newcommand{\CommentTok}[1]{\textcolor[rgb]{0.56,0.35,0.01}{\textit{#1}}}
\newcommand{\CommentVarTok}[1]{\textcolor[rgb]{0.56,0.35,0.01}{\textbf{\textit{#1}}}}
\newcommand{\ConstantTok}[1]{\textcolor[rgb]{0.00,0.00,0.00}{#1}}
\newcommand{\ControlFlowTok}[1]{\textcolor[rgb]{0.13,0.29,0.53}{\textbf{#1}}}
\newcommand{\DataTypeTok}[1]{\textcolor[rgb]{0.13,0.29,0.53}{#1}}
\newcommand{\DecValTok}[1]{\textcolor[rgb]{0.00,0.00,0.81}{#1}}
\newcommand{\DocumentationTok}[1]{\textcolor[rgb]{0.56,0.35,0.01}{\textbf{\textit{#1}}}}
\newcommand{\ErrorTok}[1]{\textcolor[rgb]{0.64,0.00,0.00}{\textbf{#1}}}
\newcommand{\ExtensionTok}[1]{#1}
\newcommand{\FloatTok}[1]{\textcolor[rgb]{0.00,0.00,0.81}{#1}}
\newcommand{\FunctionTok}[1]{\textcolor[rgb]{0.00,0.00,0.00}{#1}}
\newcommand{\ImportTok}[1]{#1}
\newcommand{\InformationTok}[1]{\textcolor[rgb]{0.56,0.35,0.01}{\textbf{\textit{#1}}}}
\newcommand{\KeywordTok}[1]{\textcolor[rgb]{0.13,0.29,0.53}{\textbf{#1}}}
\newcommand{\NormalTok}[1]{#1}
\newcommand{\OperatorTok}[1]{\textcolor[rgb]{0.81,0.36,0.00}{\textbf{#1}}}
\newcommand{\OtherTok}[1]{\textcolor[rgb]{0.56,0.35,0.01}{#1}}
\newcommand{\PreprocessorTok}[1]{\textcolor[rgb]{0.56,0.35,0.01}{\textit{#1}}}
\newcommand{\RegionMarkerTok}[1]{#1}
\newcommand{\SpecialCharTok}[1]{\textcolor[rgb]{0.00,0.00,0.00}{#1}}
\newcommand{\SpecialStringTok}[1]{\textcolor[rgb]{0.31,0.60,0.02}{#1}}
\newcommand{\StringTok}[1]{\textcolor[rgb]{0.31,0.60,0.02}{#1}}
\newcommand{\VariableTok}[1]{\textcolor[rgb]{0.00,0.00,0.00}{#1}}
\newcommand{\VerbatimStringTok}[1]{\textcolor[rgb]{0.31,0.60,0.02}{#1}}
\newcommand{\WarningTok}[1]{\textcolor[rgb]{0.56,0.35,0.01}{\textbf{\textit{#1}}}}
\usepackage{graphicx,grffile}
\makeatletter
\def\maxwidth{\ifdim\Gin@nat@width>\linewidth\linewidth\else\Gin@nat@width\fi}
\def\maxheight{\ifdim\Gin@nat@height>\textheight\textheight\else\Gin@nat@height\fi}
\makeatother
% Scale images if necessary, so that they will not overflow the page
% margins by default, and it is still possible to overwrite the defaults
% using explicit options in \includegraphics[width, height, ...]{}
\setkeys{Gin}{width=\maxwidth,height=\maxheight,keepaspectratio}
% Set default figure placement to htbp
\makeatletter
\def\fps@figure{htbp}
\makeatother
\setlength{\emergencystretch}{3em} % prevent overfull lines
\providecommand{\tightlist}{%
  \setlength{\itemsep}{0pt}\setlength{\parskip}{0pt}}
\setcounter{secnumdepth}{-\maxdimen} % remove section numbering

\title{Tarea 3: LaTeX y RMarkdown}
\author{\(\mathcal Imanol\ Miguez\)}
\date{11/2/2021}

\begin{document}
\maketitle

\hypertarget{preguntas}{%
\section{Preguntas}\label{preguntas}}

\hypertarget{pregunta-1}{%
\subsection{Pregunta 1}\label{pregunta-1}}

Realizad los siguientes productos de matrices siguiente en \(R\):
\[A\cdot B\] \[B\cdot A\] \[(A\cdot B)^t\] \[B^t\cdot B\]
\[(A\cdot B)^{-1}\] donde
\[A = \begin{pmatrix} 1 & 2 & 3 & 4\\ 4 & 3 & 2 & 1\\ 0 & 1 & 0 & 2\\ 3 & 0 & 4 & 0 \end{pmatrix}\ \ \ \ \ B = \begin{pmatrix} 4 & 3 & 2 & 1\\ 0 & 3 & 0 & 4\\ 1 & 2 & 3 & 4\\ 0 & 1 & 0 & 2 \end{pmatrix}\]

Finalmente, escribe haciendo uso de \LaTeX~el resultado de los dos
primeros productos de forma adecuada.

Primero creamos las matrices mediante la sucesion de filas de vectores.

\begin{Shaded}
\begin{Highlighting}[]
\NormalTok{A =}\StringTok{ }\KeywordTok{rbind}\NormalTok{(}\KeywordTok{c}\NormalTok{(}\DecValTok{1}\NormalTok{,}\DecValTok{2}\NormalTok{,}\DecValTok{3}\NormalTok{,}\DecValTok{4}\NormalTok{),}\KeywordTok{c}\NormalTok{(}\DecValTok{4}\NormalTok{,}\DecValTok{3}\NormalTok{,}\DecValTok{2}\NormalTok{,}\DecValTok{1}\NormalTok{),}\KeywordTok{c}\NormalTok{(}\DecValTok{0}\NormalTok{,}\DecValTok{1}\NormalTok{,}\DecValTok{0}\NormalTok{,}\DecValTok{2}\NormalTok{),}\KeywordTok{c}\NormalTok{(}\DecValTok{3}\NormalTok{,}\DecValTok{0}\NormalTok{,}\DecValTok{4}\NormalTok{,}\DecValTok{0}\NormalTok{))}
\NormalTok{B =}\StringTok{ }\KeywordTok{rbind}\NormalTok{(}\KeywordTok{c}\NormalTok{(}\DecValTok{4}\NormalTok{,}\DecValTok{3}\NormalTok{,}\DecValTok{2}\NormalTok{,}\DecValTok{1}\NormalTok{),}\KeywordTok{c}\NormalTok{(}\DecValTok{0}\NormalTok{,}\DecValTok{3}\NormalTok{,}\DecValTok{0}\NormalTok{,}\DecValTok{4}\NormalTok{),}\KeywordTok{c}\NormalTok{(}\DecValTok{1}\NormalTok{,}\DecValTok{2}\NormalTok{,}\DecValTok{3}\NormalTok{,}\DecValTok{4}\NormalTok{),}\KeywordTok{c}\NormalTok{(}\DecValTok{0}\NormalTok{,}\DecValTok{1}\NormalTok{,}\DecValTok{0}\NormalTok{,}\DecValTok{2}\NormalTok{))}
\end{Highlighting}
\end{Shaded}

Despues realizamos las operaciones solicitadas. Para las
multiplicaciones entre matrices usamos \(\%*\%\), para la traspuesta
usamos \(t(M)\) y para la inversa \(solve(M)\).

\begin{Shaded}
\begin{Highlighting}[]
\NormalTok{M1 =}\StringTok{ }\NormalTok{A}\OperatorTok\NormalTok{B}
\NormalTok{M2 =}\StringTok{ }\NormalTok{B}\OperatorTok\NormalTok{A}
\KeywordTok{t}\NormalTok{(A}\OperatorTok\NormalTok{B)}
\end{Highlighting}
\end{Shaded}

\begin{verbatim}
##      [,1] [,2] [,3] [,4]
## [1,]    7   18    0   16
## [2,]   19   26    5   17
## [3,]   11   14    0   18
## [4,]   29   26    8   19
\end{verbatim}

\begin{Shaded}
\begin{Highlighting}[]
\KeywordTok{t}\NormalTok{(B)}\OperatorTok\NormalTok{A}
\end{Highlighting}
\end{Shaded}

\begin{verbatim}
##      [,1] [,2] [,3] [,4]
## [1,]    4    9   12   18
## [2,]   18   17   19   19
## [3,]    2    7    6   14
## [4,]   23   18   19   16
\end{verbatim}

\begin{Shaded}
\begin{Highlighting}[]
\KeywordTok{solve}\NormalTok{(A}\OperatorTok\NormalTok{B)}
\end{Highlighting}
\end{Shaded}

\begin{verbatim}
##       [,1]  [,2]  [,3]  [,4]
## [1,] -1.66 -0.65  4.52  1.52
## [2,]  1.60  0.80 -4.60 -1.60
## [3,]  1.02  0.35 -2.84 -0.84
## [4,] -1.00 -0.50  3.00  1.00
\end{verbatim}

\begin{Shaded}
\begin{Highlighting}[]
\KeywordTok{solve}\NormalTok{(A)}\OperatorTok\KeywordTok{t}\NormalTok{(B)}
\end{Highlighting}
\end{Shaded}

\begin{verbatim}
##               [,1] [,2] [,3] [,4]
## [1,]  6.000000e-01  2.4  6.4  1.2
## [2,] -3.330669e-16 -2.0 -7.0 -1.2
## [3,] -2.000000e-01 -0.8 -3.8 -0.4
## [4,]  1.000000e+00  1.0  5.0  0.6
\end{verbatim}

Los resultados son: - \$A\cdot B \$ 7, 18, 0, 16, 19, 26, 5, 17, 11, 14,
0, 18, 29, 26, 8, 19

\hypertarget{pregunta-2}{%
\subsection{Pregunta 2}\label{pregunta-2}}

Considerad en un vector los números de vuestro \(DNI\) y llamadlo
\textbf{dni}. Por ejemplo, si vuestro \(DNI\) es 54201567K, vuestro
vector será \[dni = (5,4,2,0,5,6,7)\]

Definid el vector en \(R\). Calculad con \(R\) el vector dni al
cuadrado, la raíz cuadrada del vector dni y, por último,la suma de todas
las cifras del vector \textbf{dni}.

Finalmente, escribid todos estos vectores también a \LaTeX

\begin{Shaded}
\begin{Highlighting}[]
\NormalTok{dni =}\StringTok{ }\KeywordTok{c}\NormalTok{(}\DecValTok{5}\NormalTok{,}\DecValTok{4}\NormalTok{,}\DecValTok{2}\NormalTok{,}\DecValTok{0}\NormalTok{,}\DecValTok{1}\NormalTok{,}\DecValTok{5}\NormalTok{,}\DecValTok{6}\NormalTok{,}\DecValTok{7}\NormalTok{)}

\NormalTok{dni}\OperatorTok{^}\DecValTok{2}
\end{Highlighting}
\end{Shaded}

\begin{verbatim}
## [1] 25 16  4  0  1 25 36 49
\end{verbatim}

\begin{Shaded}
\begin{Highlighting}[]
\KeywordTok{sqrt}\NormalTok{(dni)}
\end{Highlighting}
\end{Shaded}

\begin{verbatim}
## [1] 2.236068 2.000000 1.414214 0.000000 1.000000 2.236068 2.449490 2.645751
\end{verbatim}

\begin{Shaded}
\begin{Highlighting}[]
\KeywordTok{sum}\NormalTok{(dni)}
\end{Highlighting}
\end{Shaded}

\begin{verbatim}
## [1] 30
\end{verbatim}

\hypertarget{pregunta-3}{%
\subsection{Pregunta 3}\label{pregunta-3}}

Considerad el vector de las letras de vuestro nombre y apellido.
Llamadlo \textbf{name}. Por ejemplo, en mi caso sería
\[nombre = (M,A,R,I,A,S,A,N,T,O,S)\]

Definid dicho vector en \(R\). Calculad el subvector que solo contenga
vuestro nombre. Calculad también el subvector que contenga solo vuestro
apellido. Ordenadlo alfabéticamente. Cread una matriz con este vector.

\begin{Shaded}
\begin{Highlighting}[]
\NormalTok{name =}\StringTok{ }\KeywordTok{c}\NormalTok{(}\StringTok{"I"}\NormalTok{,}\StringTok{"M"}\NormalTok{,}\StringTok{"A"}\NormalTok{,}\StringTok{"N"}\NormalTok{,}\StringTok{"O"}\NormalTok{,}\StringTok{"L"}\NormalTok{,}\StringTok{"M"}\NormalTok{,}\StringTok{"I"}\NormalTok{,}\StringTok{"G"}\NormalTok{,}\StringTok{"U"}\NormalTok{,}\StringTok{"E"}\NormalTok{,}\StringTok{"Z"}\NormalTok{)}
\NormalTok{name[}\DecValTok{1}\OperatorTok{:}\DecValTok{6}\NormalTok{]}
\end{Highlighting}
\end{Shaded}

\begin{verbatim}
## [1] "I" "M" "A" "N" "O" "L"
\end{verbatim}

\begin{Shaded}
\begin{Highlighting}[]
\NormalTok{name[}\DecValTok{7}\OperatorTok{:}\DecValTok{12}\NormalTok{]}
\end{Highlighting}
\end{Shaded}

\begin{verbatim}
## [1] "M" "I" "G" "U" "E" "Z"
\end{verbatim}

\begin{Shaded}
\begin{Highlighting}[]
\KeywordTok{sort}\NormalTok{(name)}
\end{Highlighting}
\end{Shaded}

\begin{verbatim}
##  [1] "A" "E" "G" "I" "I" "L" "M" "M" "N" "O" "U" "Z"
\end{verbatim}

\begin{Shaded}
\begin{Highlighting}[]
\KeywordTok{rbind}\NormalTok{(name)}
\end{Highlighting}
\end{Shaded}

\begin{verbatim}
##      [,1] [,2] [,3] [,4] [,5] [,6] [,7] [,8] [,9] [,10] [,11] [,12]
## name "I"  "M"  "A"  "N"  "O"  "L"  "M"  "I"  "G"  "U"   "E"   "Z"
\end{verbatim}

\end{document}
